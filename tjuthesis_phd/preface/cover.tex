% !Mode:: "TeX:UTF-8"

\cheading{天津大学博士学位论文}
\ctitle{超高速反重力昆虫飞行器的研究}  %封面用论文标题,自己可手动断行
\etitle{A Study about Super High Speed Anti-gravity Insect Aircraft}
\caffil{天津大学计算机科学与电气工程学院} %学院名称
\cmacrosubjecttitle{一级学科}
\cmacrosubject{控制科学与工程}
\csubjecttitle{学科专业}
\csubject{模式识别与智能系统}   %专业
\cauthortitle{研究生}     % 学位
\cauthor{Jack.Peiyang}   %学生姓名
\csupervisortitle{指导教师}
\csupervisor{西格尔教授(美)} %导师姓名

\declaretitle{独创性声明}
\declarecontent{
本人声明所呈交的学位论文是本人在导师指导下进行的研究工作和取得的研究成果,除了文中特别加以标注和致谢之处外,论文中不包含其他人已经发表或撰写过的研究成果,也不包含为获得 {\underline{\kai\textbf{~天津大学~}}}或其他教育机构的学位或证书而使用过的材料。与我一同工作的同志对本研究所做的任何贡献均已在论文中作了明确的说明并表示了谢意。
}
\authorizationtitle{学位论文版权使用授权书}
\authorizationcontent{
本学位论文作者完全了解{\underline{\kai\textbf{~天津大学~}}}有关保留、使用学位论文的规定。特授权{\underline{\kai\textbf{~天津大学~}}} 可以将学位论文的全部或部分内容编入有关数据库进行检索,并采用影印、缩印或扫描等复制手段保存、汇编以供查阅和借阅。同意学校向国家有关部门或机构送交论文的复印件和磁盘。
}
\authorizationadd{(保密的学位论文在解密后适用本授权说明)}
\authorsigncap{学位论文作者签名:}
\supervisorsigncap{导师签名:}
\signdatecap{签字日期:}


\cdate{\CJKdigits{\the\year} 年\CJKnumber{\the\month} 月 \CJKnumber{\the\day} 日}
% 如需改成二零一二年四月二十五日的格式,可以直接输入,即如下所示
% \cdate{二零一二年四月二十五日}
%\cdate{\the\year 年\the\month 月 \the\day 日} % 此日期显示格式为阿拉伯数字 如2012年4月25日
\cabstract{
中文摘要应将学位论文的内容要点简短明了地表达出来,约500~800字左右(限一页),字体为宋体小四号。内容应包括工作目的、研究方法、成果和结论。要突出本论文的创新点,语言力求精炼。为了便于文献检索,应在本页下方另起一行注明论文的关键词(3-7个)。

天津大学是教育部直属国家重点大学,其前身为北洋大学,始建于1895年10月2日,是中国第一所现代大学,素以“实事求是”的校训、“严谨治学”的校风和“爱国奉献”的传统享誉海内外。1951年经国家院系调整定名为天津大学,是1959年中共中央首批确定的16所国家重点大学之一,是“211工程”、“985工程”首批重点建设的大学,入选国家“世界一流大学建设”A 类高校。

2018年,天津大学分别建立了智能和计算学部和医学院。

}

\ckeywords{[关键词1]  [关键词2]  [关键词3]  [关键词4] }

\eabstract{
Externally pressurized gas bearing has been widely used in the field of aviation, semiconductor, weave, and measurement apparatus because of its advantage of high accuracy, little friction, low heat distortion, long life-span, and no pollution. In this thesis, based on the domestic and overseas researching……

Tianjin University is the first institution of higher education in China, pioneering the development of modern Chinese higher education. Its history can be traced back to Peiyang University, which was founded in October 2, 1895, and renamed Tianjin University in September 1951 with the approval of the Government Administration Council of the Central People’s Government during the nationwide education restructuring of colleges and departments. In 1959, Tianjin University was appointed as one of the National Key Universities by the Central Committee of the Communist Party of China,and in 2000 was selected as one of the high-level research universities to be supported by Project 985.

In 2018, Tianjin University established College of Intelligence and Computing, and School of Medical Science and Engineering. 
}

\ekeywords{"[Key Word 1]" , "[Key Word 2]" , "[Key Word 3]" , "[Key Word 4]" }

\makecover

\clearpage
